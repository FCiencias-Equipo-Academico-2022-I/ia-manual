\subsection{feed.py}
\label{appe:auxiliar}

\begin{minted}[mathescape,
               linenos,
               numbersep=5pt,
               gobble=2,
               frame=lines,
               fontsize=\scriptsize,
               framesep=2mm]{python}
    # -*- coding: utf-8 -*-

    import feedparser
    import urllib3
    import re

    url = 'http://aristeguinoticias.com/feed/'
    db = './feed.db'

    feed = feedparser.parse(url)
    not_in_db = True

    def cleanhtml(raw_html):
        """
        Removes any strange symbol from a HTML string
        """
        cleanr = re.compile('<.*?>')
        cleantext = re.sub(cleanr, '', raw_html)
        return cleantext

    def cleanstr(raw_str):
        """
        Removes any strange symbol from a string
        """
        cleanr = re.compile('&.*?;')
        cleantext = re.sub(cleanr, '', raw_str)
        return cleantext


    for post in feed.entries:
        title = post.title
        link = post.link

        with open(db) as database:
            for line in database:
                if title in line:
                    # Verifying in the entry already exists in the database
                    print('Already in DB')
                    not_in_db = False
                    break

        if not not_in_db:
            not_in_db = True
            continue

        try:
            # Fetching the news data
            http = urllib3.PoolManager()
            r = http.request('GET', post.link)
            data = str(r.data, 'utf-8')
            text = data.split('<div class="class_text">')[1]
            text = text.split('</div>')[0]
            text = cleanhtml(text)
        except Exception as e:
            continue

        summary = cleanstr(text.strip()).replace('\n', ' ')
        tags = map(lambda x: x.term,post.tags)
        tags_clean = ''

        for tag in tags:
            if tag != 'LO + DESTACADO':
                tags_clean = tags_clean + tag + ', '

        tags_clean = tags_clean[:len(tags_clean)-2]

        print("Title: " + title)
        print("Tags: " + tags_clean)
        print("Text: " + summary)
        print("Link: " + link)
        print('')

        with open(db, 'a') as file:
            file.write("Title: " + title + '\n')
            file.write("Tags: " + tags_clean + '\n')
            file.write("Text: " + summary + '\n')
            file.write("Link: " + link + '\n\n')

\end{minted}

\subsection{feed.db}
\label{appe:base}

Si se necesita este archivo, favor de mandar un correo a \href{mailto:luis.lzrg@gmail.com}{luis.lzrg@gmail.com}, ya que por ser un archivo muy extenso decidí no agregarlo.