%----------------------------------------------------------------------------------------
% Cadenas de Márkov
%----------------------------------------------------------------------------------------

En esta práctica se implementarán funciones para realizar cálculos básicos sobre cadenas de Márkov.

\section{Objetivo}

\begin{compactitem}
 \item Que el alumno se familiarice con los procesos estocásticos discretos denominados Cadenas de Márkov discretas y realice una implementación para resolver algunas de las preguntas más frecuentes que se pueden plantear sobre estos procesos.
\end{compactitem}


\section{Introducción}

Adicionalmente a lo visto en clase, que estuvo basado en el sitio de \hrefformat{https://www.gestiondeoperaciones.net/cadenas-de-markov/cadenas-de-markov-ejercicios-resueltos/}{Gestión de operaciones}, pueden consultar la página sobre cadenas de Márkov en \hrefformat{https://en.wikipedia.org/wiki/Markov_chain}{Wikipedia} así como algunos \hrefformat{https://en.wikipedia.org/wiki/Examples_of_Markov_chains}{ejemplos} sencillos.


\section{Desarrollo}

Para esta práctica se requiere implementar una clase para trabajar con cadenas de Márkov.  Se recomienda programarla en \code{Python}, ya que la biblioteca \code{numpy} contiene métodos para trabajar con matrices y vectores, lo cual simplifica mucho los puntos siguientes.  También se recomienda utilizar matrices columna para representar vectores, de esta manera funcionarán correctamente las multiplicaciones de matriz por vector.

La clase \code{CadenaDeMarkov} debe contener:

\begin{enumerate}
 \item Un constructor que reciba como parámetros:
 \begin{enumerate}
  \item Una lista con los nombres los estados posibles.
  \item Un vector con la probabilidad de iniciar en cada uno de los estados posibles.
  \item Una matriz de probabilidades, con la probabilidad de transitar de cada estado hacia los demás.
 \end{enumerate}

 \item Un método para generar una secuencia de estados a partir del modelo de Márkov iniciado dado el número $n$ de elementos que tendrá la secuencia; opcionalmente puede recibir como parámetro una semilla para la generación de números aleatorios.
 
 Para generar esta muestra necesitarás calcular los vectores con las distribuciones de probabilidad para $n$ pasos.  Dada cada distribución, utiliza un número aleatorio para determinar cuál de los estados corresponderá a ese paso.
 
 Devuelve una lista con la secuencia de estados.
 
 \item Obtener la probabilidad de una cadena de estados (punto extra si se permite que esta cadena tenga estados indeterminados).  Observa que esta es la distribución de probabilidad conjunta $P(S_0=s_0,S_1=s_1,...,S_n=s_n)$, escrito de otra forma, $P(s_0,s_1,...,s_n)$.  Deberá recibir como parámetro una lista con la secuencia de estados y devolver la probabilidad.
 
 \item Estimar las probabilidades a largo plazo de cada uno de los estados, es decir, la distribución límite, cuando sea posible.  Este método deberá devolver el vector con la distribución de probabilidades.
 
 \item Agregar un archivo donde se utilice tu clase para resolver un ejemplo, usando cada uno de los métodos.  Puedes usar algún ejemplo del sitio de gestión de operaciones.
\end{enumerate}


\subsection{Requisitos y resultados}

Incluir:

\begin{enumerate}
 \item El archivo de la clase \code{CadenaDeMarkov}.
 \item El archivo con el demo de prueba.
\end{enumerate}

